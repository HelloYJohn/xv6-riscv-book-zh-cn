
\documentclass[UTF8]{article}
\usepackage{xeCJK}
\usepackage{amsmath,amssymb}
\begin{document}

   \chapter{Summary}   
    \label{CH:SUM}     

本文通过对操作系统xv6的逐行研究,介绍了操作系统的主要思想。有些代码行体现了主要思想的本质(例如,上下文切换、用户/内核边界、锁等),并且每一行都很重要;其他代码行提供了如何实现特定操作系统想法的说明,并且可以通过不同的方式轻松完成(例如,更好的调度算法、更好的磁盘数据结构来表示文件、更好的日志记录以允许并发事务等)。 )。所有的想法都是在一个特定的、非常成功的系统调用接口(Unix 接口)的背景下阐述的,但这些想法也延续到了其他操作系统的设计中。  


  

\end{document}

