

\chapter*{Foreword and acknowledgments}     

这是为操作系统课程准备的文本草稿。它通过研究一个名为 xv6 的示例内核来解释操作系统的主要概念。 Xv6 仿照 Dennis Ritchie 和 Ken Thompson 的 Unix Version 6 (v6)~    \cite{unix}    。 Xv6 大致遵循 v6 的结构和风格,但在多核 RISC-V~    \cite{riscv}    的 ANSI C~    \cite{kernighan}    中实现。  

本文应与 xv6 的源代码一起阅读,该方法的灵感来自 John Lions 的 UNIX 第六版评论 ~    \cite{lions}    。看
    \url{https://pdos.csail.mit.edu/6.1810}    用于指向 v6 和 xv6 的在线资源,包括使用 xv6 的多个实验室作业。  

我们在 6.828 和 6.1810(麻省理工学院的操作系统类)中使用了此文本。我们感谢那些直接或间接为 xv6 做出贡献的教师、助教和学生。我们特别要感谢 Adam Belay、Austin Clements 和 Nickolai Zeldovich。最后,我们要感谢通过电子邮件向我们发送文本中的错误或改进建议的人:Abutalib Aghayev, Sebastian Boehm, brandb97, Anton
Burtsev, Raphael Carvalho, Tej Chajed, Rasit Eskicioglu, Color Fuzzy,
Wojciech Gac, Giuseppe, Tao Guo, Haibo Hao, Naoki Hayama, Chris
Henderson, Robert Hilderman, Eden Hochbaum, Wolfgang Keller, Henry
Laih, Jin Li, Austin Liew, Pavan Maddamsetti, Jacek Masiulaniec,
Michael McConville, m3hm00d, miguelgvieira, Mark Morrissey, Muhammed
Mourad, Harry Pan, Harry Porter, Siyuan Qian, Askar Safin, Salman
Shah, Huang Sha, Vikram Shenoy,  Adeodato Simó, Ruslan Savchenko, Pawel Szczurko,
Warren Toomey, tyfkda, tzerbib, Vanush Vaswani, Xi Wang, and Zou Chang
Wei, Sam Whitlock, LucyShawYang, and Meng Zhou  

如果您发现错误或有改进建议,请发送电子邮件至 Frans Kaashoek 和 Robert Morris (kaashoek,rtm@csail.mit.edu)。  


